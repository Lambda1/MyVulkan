\documentclass{suribt}

\usepackage[dvipdfmx]{graphicx}

\usepackage{here}
\usepackage{mathtools}
\usepackage{bm}
\usepackage{siunitx}
\usepackage{url}

\usepackage{listings,jlisting}

\renewcommand{\lstlistingname}{リスト}
\lstset{
	language=c,
	basicstyle=\ttfamily\scriptsize,
	commentstyle=\textit,
	classoffset=1,
	keywordstyle=\bfseries,
	frame=tRBl,
	framesep=5pt,
	showstringspaces=false,
	numbers=left,
	stepnumber=1,
	numberstyle=\tiny,
	tabsize=2
}

\title{VulkanやCGに関するメモ}
\author{Lambda1}
\supervisor{}
\studentid{}
\handin{2020}{03}
\keywords{Vulkan, Computer Graphics}

\begin{document}
\maketitle
\frontmatter

\begin{abstract}
	色々なメモ
\end{abstract}

\tableofcontents

\mainmatter
\chapter{はじめに}\label{chap:1}
\section{Vulkanとは}
	Khronos GroupがSIGGRAPH2014で発表した「クロスプラットフォーム3Dコンピュータグラフィックス・コンピュートAPI」であり, ハードウェアの限界性能を引き出すために開発されたローレベルAPIである.
\subsection{特徴}
\begin{itemize}
	\item プログラマが意図した動作を明確(explicit)に指定することでドライバ・オーバヘッドの削減
	\item コマンド作成とサブミットをマルチスレッドで並列化
	\item SPIR-Vによって, シェーダプログラムをバイトコードに変換することで編集を効率化(SPIR-Vで吐き出された中間言語ならGLSL, HLSLとかなんでもおk)
\end{itemize}
\newpage
\section{三角形描画までの道のり}
	いわゆるグラフィック版「hello, world!」.
	基本的な詳細はVulkan Tutorialを参照\cite{1-a}.
\begin{enumerate}
	\item Instance and physical device selection\\
		1. インスタンス(VkInstance)作成\\
		2. サポートしているグラフィックカード(VkPhysicalDevice)選択
	\item Logical device and queue families\\
		描画のために論理デバイス(VkDevice)とキューファミリ(VkQueue)作成
	\item Window surface and swap chain\\
		ウィンドウサーフェスとスワップチェーン作成
	\item Image views and framebuffers\\
		スワップチェーンイメージをVkImageView内にラップ
	\item Render passes\\
		レンダーターゲットとレンダーパス作成
	\item Graphics pipeline\\
		グラフィックスパイプラインを構築
	\item Command pools and command buffers\\
		利用可能なスワップチェーンイメージにドローコマンドのコマンドバッファの割り当てと記録
	\item main loop\\
		取得済みイメージからバッファの描画, 正しいドローコマンドバッファのサブミット, スワップチェーンにイメージ返却
\end{enumerate}
\newpage

\chapter{CG 用語集}\label{chap:2}
\section{A}

\newpage
\chapter{Vulkan 用語集}\label{chap:3}
\section{vkCreate(name)関数}
	Vulkanオブジェクト生成のための生成関数.
	パラメータ設定したVK(name)CreateInfo構造体を引数として使用する.
\section{Vk(name)CreateInfo構造体}
	オブジェクト設定を行うための構造体.

\newpage
\chapter{ライブラリ 用語集}\label{chap:3}
\section{GLFW3}

\backmatter

\begin{thebibliography}{}
\bibitem{1-a}
	Alexander Overvoorde,
	``Vulkan Tutorial'',\\
	\url{https://vulkan-tutorial.com}, 2020/02/25(閲覧).
\end{thebibliography}

\appendix
\chapter{プログラムリスト}
\section{モデル}

\end{document}
